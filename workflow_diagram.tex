\documentclass[tikz,border=10pt]{standalone}
\usepackage{tikz}
\usetikzlibrary{shapes.geometric, arrows.meta, positioning, fit, backgrounds, calc, shapes.multipart}

% Define colors
\definecolor{stepcolor1}{RGB}{76, 175, 80}      % Green
\definecolor{stepcolor2}{RGB}{33, 150, 243}     % Blue
\definecolor{stepcolor3}{RGB}{255, 152, 0}      % Orange
\definecolor{stepcolor4}{RGB}{156, 39, 176}     % Purple
\definecolor{stepcolor5}{RGB}{0, 150, 136}      % Teal
\definecolor{decisioncolor}{RGB}{255, 193, 7}   % Amber

\tikzset{
    step/.style={
        rectangle,
        rounded corners,
        minimum width=4.5cm,
        minimum height=1.2cm,
        text centered,
        draw=black,
        line width=1pt,
        font=\sffamily\small,
        text width=4cm,
        align=center
    },
    decision/.style={
        diamond,
        aspect=2,
        minimum width=3cm,
        minimum height=1cm,
        text centered,
        draw=black,
        line width=1pt,
        font=\sffamily\small,
        fill=decisioncolor!30,
        text width=2.5cm,
        align=center
    },
    process/.style={
        rectangle,
        minimum width=5cm,
        minimum height=1cm,
        text centered,
        draw=black,
        line width=0.8pt,
        font=\sffamily\scriptsize,
        text width=4.5cm,
        align=left,
        fill=white
    },
    arrow/.style={
        -Stealth,
        line width=1.5pt,
        draw=black!80
    },
    label/.style={
        font=\sffamily\bfseries\footnotesize,
        text=black!70
    },
    time/.style={
        font=\sffamily\scriptsize\itshape,
        text=blue!70
    }
}

\begin{document}

\begin{tikzpicture}[node distance=1.3cm]

    % Title
    \node[font=\sffamily\bfseries\Large] at (0, 11) {Agentic Workflow Execution Flow};

    % Step 1: User Request
    \node[step, fill=stepcolor1!20] (step1) at (0, 9.2) {
        \textbf{1. User Request}\\
        Natural language input
    };
    \node[process, below=0.2cm of step1] (step1_detail) {
        Example: "Run DFT calculation\\
        on ethanol (CCO) for catalyst design"
    };

    % Step 2: Agent Parsing
    \node[step, fill=stepcolor2!20, below=1.8cm of step1_detail] (step2) {
        \textbf{2. LLM Agent Parsing}\\
        Function calling decision
    };
    \node[process, below=0.2cm of step2] (step2_detail) {
        • Parse molecule (CCO = ethanol)\\
        • Identify task (DFT electronic structure)\\
        • Select tool (run\_gpaw or run\_cp2k)\\
        • Set parameters (PBE functional, etc.)
    };
    \node[time, right=0.2cm of step2.east, anchor=west] {2-5s};

    % Decision: Which Application?
    \node[decision, below=1.8cm of step2_detail] (decision) {
        Select\\
        Application
    };

    % Step 3: Job Submission
    \node[step, fill=stepcolor3!20, below=1.5cm of decision] (step3) {
        \textbf{3. FastAPI Job Submission}\\
        Immediate job\_id return
    };
    \node[process, below=0.2cm of step3] (step3_detail) {
        POST /submit\_app\_job\\
        \{application: "gpaw", params: ...\}\\
        → Returns: \{job\_id: "abc-123", status: "QUEUED"\}
    };
    \node[time, right=0.2cm of step3.east, anchor=west] {$<$100ms};

    % Step 4: Celery Processing
    \node[step, fill=stepcolor4!20, below=1.8cm of step3_detail] (step4) {
        \textbf{4. Celery Async Execution}\\
        Worker processes job
    };
    \node[process, below=0.2cm of step4] (step4_detail) {
        • Wrapper generates input files\\
        • Executes application (GPAW/CP2K/etc.)\\
        • Monitors progress\\
        • Parses output files\\
        • Extracts scientific results
    };
    \node[time, right=0.2cm of step4.east, anchor=west] {10s-30min};

    % Step 5: Agent Polling
    \node[step, fill=stepcolor5!20, below=1.8cm of step4_detail] (step5) {
        \textbf{5. Agent Status Polling}\\
        check\_simulation\_status()
    };
    \node[process, below=0.2cm of step5] (step5_detail) {
        Loop every 2 seconds:\\
        GET /job\_status/\{job\_id\}\\
        Until status == "SUCCESS"
    };

    % Decision: Job Complete?
    \node[decision, below=1.5cm of step5_detail] (decision2) {
        Job\\
        Complete?
    };

    % Step 6: Result Interpretation
    \node[step, fill=stepcolor1!20, below=1.5cm of decision2] (step6) {
        \textbf{6. Agent Result Interpretation}\\
        Scientific analysis
    };
    \node[process, below=0.2cm of step6] (step6_detail) {
        • Energy: -0.537 Ha (-14.61 eV)\\
        • Convergence: Achieved\\
        • HOMO-LUMO gap: 4.82 eV\\
        • Professional report generation
    };
    \node[time, right=0.2cm of step6.east, anchor=west] {2-5s};

    % Step 7: User Response
    \node[step, fill=stepcolor2!20, below=1.8cm of step6_detail] (step7) {
        \textbf{7. Human-Readable Response}\\
        Formatted scientific summary
    };

    % Arrows
    \draw[arrow] (step1_detail) -- (step2);
    \draw[arrow] (step2_detail) -- (decision);
    \draw[arrow] (decision) -- node[label, right] {Chosen} (step3);
    \draw[arrow] (step3_detail) -- (step4);
    \draw[arrow] (step4_detail) -- (step5);
    \draw[arrow] (step5_detail) -- (decision2);

    % Loop back arrow
    \draw[arrow, dashed] (decision2.west) -- ++(-2,0)
        node[label, above, pos=0.5] {No: Continue polling}
        |- (step5.west);

    \draw[arrow] (decision2) -- node[label, right] {Yes} (step6);
    \draw[arrow] (step6_detail) -- (step7);

    % Application options (branching from decision)
    \node[process, fill=blue!10, left=2cm of decision, text width=2cm,
          minimum width=2.3cm, font=\sffamily\tiny] (app_list) {
        \textbf{Options:}\\
        • Quantum ESPRESSO\\
        • CP2K\\
        • GPAW\\
        • LAMMPS\\
        • GROMACS
    };
    \draw[arrow, dashed, line width=1pt] (app_list) -- (decision);

    % Key Benefits Box
    \node[process, fill=green!10, draw=green!60, line width=1pt,
          right=1.8cm of step4, anchor=west, text width=3cm, minimum width=3.2cm,
          font=\sffamily\tiny] (benefits) {
        \textbf{KEY BENEFITS:}\\[0.1cm]
        \textbullet\ No LLM timeout\\
        (Step 3 returns immediately)\\[0.1cm]
        \textbullet\ Async processing\\
        (Step 4 runs independently)\\[0.1cm]
        \textbullet\ Progress monitoring\\
        (Step 5 polls status)\\[0.1cm]
        \textbullet\ Autonomous execution\\
        (No human intervention)
    };

    % Total Time Box
    \node[process, fill=yellow!15, draw=orange!60, line width=1pt,
          below=0.5cm of benefits, text width=3cm, minimum width=3.2cm,
          font=\sffamily\tiny, align=center] {
        \textbf{TYPICAL TOTAL TIME}\\[0.1cm]
        DFT: 30s - 5 min\\
        Classical MD: 10s - 2 min\\[0.1cm]
        Agent overhead: $<$10s\\
        Simulation: 10s - 30 min
    };

    % Success Metrics Box
    \node[process, fill=blue!10, draw=blue!60, line width=1pt,
          left=1.8cm of step5, anchor=east, text width=3cm, minimum width=3.2cm,
          font=\sffamily\tiny] (metrics) {
        \textbf{TESTED METRICS:}\\[0.1cm]
        \textbullet\ 100\% success rate\\
        \textbullet\ 5/5 applications working\\
        \textbullet\ 3 LLM models tested\\
        \textbullet\ gpt-oss:120b: Outstanding\\[0.1cm]
        \textbf{Test molecule:}\\
        Caffeine (C8H10N4O2)\\
        24 atoms
    };

    % Timeline arrow on right side
    \draw[-Stealth, line width=2pt, draw=gray!50]
        ($(step1.east) + (7.5, 0.5)$) --
        ($(step7.east) + (7.5, -0.5)$)
        node[midway, right, font=\sffamily\footnotesize, text=gray!70, align=center]
        {Time\\Flow};

\end{tikzpicture}

\end{document}
